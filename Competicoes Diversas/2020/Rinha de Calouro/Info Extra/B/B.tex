\section{Problema B: Super Feijões}
\vspace{-0.52cm}
\noindent \begin{verbatim}Arquivo: B.[c|cpp|java|py]                           Tempo limite: 1 s
\end{verbatim}

Omar é fã de shows de mágica, ele já foi em tantos que conhece todos os truques. Porém ele sempre se perde quando o mágico resolve fazer o truque dos feijões, no qual ele coloca um feijão embaixo de um dos quatro copos e os embaralha para perguntar à plateia onde o feijão foi parar. \\

Por isso ele resolveu pedir sua ajuda pra criar um programa que dado a posição inicial do feijão e todas as trocas realizadas, diga em qual copo o feijão está no final do truque. \\

\subsection*{Entrada}

Na primeira linha seguirão quatro inteiros, $C1$, $C2$, $C3$ e $C4$ separados por um espaço. O valor $Ci=1$ indica que o feijão estava no copo na posição $i$, e $Ci=0$ indica que o $i$-ésimo copo esta vazio. \\
Em seguida, segue um número $N$, o número de trocas realizadas. As próximas $N$ linhas contêm dois inteiros $i, j$, representando que o $i$-ésimo copo foi trocado de lugar com o $j$-ésimo copo. \\
Haverá sempre exatamente um copo com o feijão. \\

\subsection*{Restrições}
\begin{itemize}
    \item $1 \le N \le 10^5$
    \item $1 \le i, j \le 4$
\end{itemize}

\subsection*{Saída}

Baseando-se nas trocas feitas, imprima um número $i$, indicando que ao final das trocas o feijão estará no $i$-ésimo copo. \\

\begin{flushleft}
\begin{tabularx}{1.01\textwidth}{ | p{6cm} | p{10cm} | }
\hline
\textbf{Exemplo de Entrada} & \textbf{Exemplo de Saída} \\
\verbatiminput{./M/sample_1.in}
&
\verbatiminput{./M/sample_1.out}
\\
\hline
\verbatiminput{./M/sample_2.in}
&
\verbatiminput{./M/sample_2.out}
\\
\hline
\end{tabularx}
\end{flushleft}
